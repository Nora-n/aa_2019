\documentclass{aa}
\usepackage{preambule}
\usepackage[varg]{txfonts}

\begin{document}
\title{Redshift evolution of the SN stretch distribution}

%\subtitle{II. An example text with infinitesimal scientific value whose title
%and subtitle may also be split}

\author{N. Nicolas\inst{1}
    \and M. Rigault\inst{2}
}

%\offprints{R. Plemmons, \email{plemmons@...}}

\institute{Université de Lyon, F-69622, Lyon, France; Université de Lyon
    1, Villeurbanne; CNRS/IN2P3, Institut de Physique des Deux Infinis, Lyon
    \and Université Clermont Auvergne, CNRS/IN2P3, Laboratoire de
    Physique de Clermont, F-63000 Clermont-Ferrand, France.
}

\date{Received 2 November 1992 / Accepted 7 January 1993}

\abstract{Type Ia supernovae (SNe Ia) allow for the construction of the Hubble
    diagram, giving us information about the Universe's expansion and its
    fondamental components, one of which is dark energy. But systematic
    uncertainties are now starting to be limiting in our ability to measure
    those parameters. In particular, the physics of SNe Ia is still mostly
    unknown, and is thought not to change in time/with the redshift.}
    {In an attempt to reduce those uncertainties, we try to find an empirical
    law describing SNe Ia's length of explosion (stretch) evolution with the
redshift.}
    {We started by getting a complete sample representing all of the stretch
        distribution that Nature can give us, before using LsSFR measurments, an
        age tracer which evolution with redshift is known, that has been shown
        to have a strong correlation with the stretch. We compare their AICc, an
    estimator of the relative quality of statistical models that includes the
number of free parameters, to determine which ones describe besto the data.}
    {Models with an evolution of the stretch with the redshift have a better
    AICc than the ones without.}
    {We find that implementing these models allows us to fit the data better
    than models without stretch evolution.}

\keywords{Cosmology -- Type Ia Supernova -- Systematic uncertainties}
\maketitle

\section{Introduction}

Type Ia supernovae (SNe Ia) are now well-known for their capacity to determine
cosmological parameters: their study led to the discovery of the accelerated
expansion of the Universe (Riess 98, Perlmutter 99) through the name of "dark
energy", and they have been used continuously for better measurments since then
(Betoule 2014). They are acquired through their lightcurves, giving the
evolution of their luminosity from the time of explosion, in different
wavelength. 3 parameters are used to describe those: an amplitude, a width
(named "stretch") and a color (magnitude difference in the B and V bands).

The simple use of those is not enough for our aim, as SNe Ia have an intrinsec
dispersion of their luminosity of $\approx \SI{0.4}{mag}$ that gives a huge
uncertainty on the determination of their distance modulus. Henceforth they are
standardized using the "brigther-slower" and "brighter-bluer" relations
(Philipps 93, Riess 96, Tripp 98) in the SALT2 algorithm (Guy 2007, 2010) that
fits the distance modulus which is expressed as
\begin{center}
    $ \displaystyle \m = m_b + \a x_1 - \b c - M$
\end{center}
with $m_B$ the logarithm of their flux, $x_1$ their stretch, $c$ their color and
$M$ their intrinsec magnitude. This relation lowered their magnitude dispersion
to $\approx \SI{0.15}{mag}$, allowing for previously mentioned accelerated
expansion to be discovered.

This Tripp estimator lies on the idea that this standardization doesn't change
with the redshift. However, Rigault 2015 showed that SNe Ia depend on their
environments, and these environments' properties evolve with the redshift. In
this Letter, we try to determine whether a stretch evolution with the redshift
allows for a better description of the collected data.

\section{Sample}
We use data from 5 different surveys: Hubble Space Telescope (HST, REF),
Supernova Nearby Factory (SNf, REF), Supernova Legacy Survey (SNLS, REF), Sloan
DSS (SDSS, REF) and Panstarr-1 (PS1, REF). 3 of which had selection effects that
we tried to remove with a statistical approach in lack of precise data
concerning the instruments' capacity to acquire fluxes. Here it is:

\section{Method}
We used LsSFR measurments which evolution with the redshift is analytically
known (references) and stretch measurments made by SNf to try an correlate these
to parameters. We fit different model, and compare their AICc.

\section{Results}
We find that every model lacking an evolution of the stretch with the redshift
is systematically worse than those that implement it.

\section{Conclusion}
Stretch evolution with the redshift is a thing. Need to see if it has an impact
on the cosmology though.


\end{document}
