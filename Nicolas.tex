\documentclass[]{aa}

\usepackage{txfonts}
\usepackage{ulem}
\usepackage{amsmath, mathtools}
\usepackage[colorlinks,breaklinks]{hyperref}
\hypersetup{linkcolor=blue,citecolor=blue,filecolor=black,urlcolor=blue}

\newcommand{\orcid}[1]{
\href{https://orcid.org/#1}{
\includegraphics[height=11pt]{General_figures/ORCIDiD_icon128x128.png}}}

\def\l{\lambda}\def\L{\Lambda}

\newcommand{\nn}[1]{\textcolor[rgb]{1.0, 0.6, 0.0}{#1}}
\newcommand{\mr}[1]{\textcolor[rgb]{0.5, 0.20, 0.3}{#1}}
\newcommand{\yc}[1]{\textcolor[RGB]{217, 22, 102}{#1}}
\newcommand{\prob}[2]{\mathcal{P}\left( #1 \mid #2\right)}

\begin{document}
\title{Redshift Evolution of the Underlying Type Ia Supernova Stretch
Distribution}

\titlerunning{Redshift Evolution of the Underlying Type Ia Supernova Stretch
Distribution}
\authorrunning{N.~Nicolas et al.}

\author{
    N. Nicolas \thanks{n.nicolas@ip2i.in2p3.fr, equal contribution} \inst{1} 
    \and M. Rigault \thanks{m.rigault@ip2i.in2p3.fr, equal contribution} \inst{1}
    \orcid{0000-0002-8121-2560}
    \and\\ Y. Copin \inst{1}
    \orcid{0000-0002-5317-7518}
    \and R. Graziani \inst{2}
    \and G. Aldering\inst{3}
    \and M. Briday\inst{1}
    \and Y.-L. Kim\inst{1}
    \orcid{0000-0002-1031-0796}
    \and J. Nordin\inst{4}
    \and Saul Perlmutter\inst{3}
    \and M. Smith\inst{1,5}
    \orcid{0000-0002-3321-1432}
}

\institute{Univ Lyon, Univ Claude Bernard Lyon 1, CNRS, IP2I Lyon / IN2P3, IMR
    5822, F-69622, Villeurbanne, France
    \and 
    Université Clermont Auvergne, CNRS/IN2P3, Laboratoire de
    Physique de Clermont, F-63000 Clermont-Ferrand, France.
    \and
    Physics Division, Lawrence Berkeley National Laboratory, 
    1 Cyclotron Road, Berkeley, CA, 94720 
    \and
    Institut fur Physik, Humboldt-Universität zu Berlin, Newtonstr. 15,
    12489 Berlin
    \and
    University of Southampton: Southampton, GB
}

\date{Submitted to A\&A the 19th of May 2020}

\abstract{The detailed nature of type Ia supernovae (SNe~Ia) remains uncertain,
    and as survey statistics increase, the question of astrophysical systematic
    uncertainties rises, notably that of SN~Ia population evolution. In this
    paper, we study the dependence with redshift of the SN~Ia \texttt{SALT2.4}
    lightcurve stretch, a purely intrinsic SN property, to probe its potential
    redshift drift. The SN stretch has been shown to strongly correlate with the
    SN environment, notably with stellar age tracers. We model the underlying
    stretch distribution as a function of redshift, using the evolution of the
    fraction of young and old SNe~Ia as predicted by \cite{rigault2020}, and
    assuming constant underlying stretch distribution for each age population
    consisting of Gaussian mixtures. We test our prediction against published
    samples cut to have marginal magnitude selection effects, so that any
    observed change is indeed of astrophysical and not observational in origin.
    In this first study, there are indications that the underlying SN~Ia stretch
    distribution is evolving as a function of redshift, and that the young/old
    drifting model is a better description of the data than any time-constant
    model, including the sample-based asymmetric distributions often used to
    correct Malmquist bias \textbf{at a significance of over 5 $\sigma$}. The
    favored underlying stretch model is the bimodal one derived from
    \cite{rigault2020}, composed of a high-stretch mode shared by both young and
    old environments, and a low-stretch mode exclusive to old environments. The
    precise impact of the redshift evolution of the SN Ia population intrinsic
    properties on cosmology remains to be studied. Yet, the astrophysical drift
    of the SN stretch distribution does affect current Malmquist bias
    corrections and thereby distances derived from SN affected by observational
selection effects. We highlight that such a bias will increase with surveys
covering increasingly larger redshift ranges, which is particularly important
for LSST.}

\keywords{Cosmology -- Type Ia Supernova -- Systematic uncertainties}
\maketitle

\section{Introduction}

Type Ia supernovae (SNe Ia) are powerful cosmological distance indicators that
enabled the discovery of the acceleration of the Universe's expansion
\citep{riess1998, perlmutter1999}. They remain today a key cosmological probe to
understand the properties of dark energy (DE) as it is the only tool able to
precisely map the recent expansion rate ($z<0.5$), when DE is driving it
\citep[e.g.][]{scolnicastro2020}. They also are key to directly measure the
Hubble Constant ($H_0$), provided one can calibrate their absolute magnitude
\citep{riess2016, freedman2019}. Interestingly, the value of $H_0$ derived when
the SNe~Ia are anchored to Cepheids \citep[the SH0ES
project,][]{riess2009, riess2016} is $\sim5\sigma$ higher than what is predicted
from cosmic microwave background (CMB) data measured by Planck assuming the
standard $\L$CDM \citep{planck2018, riess2019, reid2019}, or when the SN
luminosity is anchored at intermediate redshift by the baryon acoustic
oscillation (BAO) scale \citep{feeney2019}. While using the tip of the red giant
branch technique in place of the Cepheids seem to favor an intermediate value of
$H_0$ \citep{freedman2019, freedman2020}, time delay measurements from strong
lensing seem to also favor high $H_0$ values \citep{wong2019}.

The $H_0$ tension has received a lot of attention, as it could be a sign of new
fundamental physics. Yet, no simple solution is able to accommodate this $H_0$
tension when accounting for all other probes \citep{knox2019}. Alternatively,
systematic effects affecting one or several of the aforementioned analyses
might explain at least some of this tension. \cite{rigault2015}
suggested that SNe Ia from the Cepheid calibrator sample differ by construction
from those in the Hubble flow sample, as the former strongly favors
young stellar populations, while the latter does not. This selection
effect would impact the derivation of $H_0$ if SNe~Ia from young and older
environments differ in average standardized magnitudes. 

The relation between SNe~Ia and their host galaxy properties has been
studied extensively. The first key finding was that the standardized SNe~Ia
magnitudes significantly depend on the host galaxy stellar mass, with
SNe~Ia from high-mass host galaxies being brighter on average
\cite[e.g.][]{kelly2010, sullivan2010, childress2013, betoule2014, rigault2020,
kim19}. This mass-step correction is currently used in cosmological analyses
\citep[e.g.][]{betoule2014, scolnic2018a}, including for deriving $H_0$
\citep{riess2016, riess2019}. Yet, the underlying connection between the SNe and
their host galaxies remains , particularlyunclear when using
global properties such as the host stellar . This in turnmass raises
the question of the accuracy of such corrections since host galaxy
global properties evolve with redshift. More recently, studies have used the
local SN host galaxy environment to probe more direct connections
between the SNe and their host galaxy environments \citep{rigault2013},
showing that local age tracers such as the Local specific Star Formation Rate
(LsSFR) or the local color are more strongly correlated with the standardized SN
magnitude \citep{roman2018, kim18,rigault2020}. Further reinforcing this
connection is the identification of SN~Ia spectral features that are correlated
with LsSFR \citep{nordin2018}. These results suggest age as the driving
parameter underlying the mass-step. If true, this would have a significant
impact for cosmology, since the environmental correction to apply to SN
standardization could strongly vary with redshift. \citep{rigault2013,
childress2014, scolnic2018a}. Yet, the importance of local SN environmental
studies remains highly debated \cite[e.g.][]{jones2015, jones2019} and
especially the impact of such an astrophysical bias has on the derivation of
$H_0$ \citep{jones2015, riess2016, riess2018, rose2019}. 

The concept of the SN~Ia age dichotomy arose with the study of the SN~Ia rate.
\cite{mannucci2005, scannapieco2005, sullivan2006, aubourg2008} have shown that
the relative SNe~Ia rate in galaxies could be explained if two populations
existed, one young, following the host star formation activity, and one old
following the host stellar mass (the so called ``prompt and delayed'' or ``A+B''
model). \cite{rigault2020} used the LsSFR to classify which are the younger
(those with a high LsSFR) and which are the older (those with low LsSFR).
Furthermore, since the first SNe~Ia host analyses, the SN stretch has been known
to be strongly correlated with the SN host galaxy properties
\citep{hamuy1996, hamuy2000}. This correlation that has been extensively
confirmed since \citep[e.g.][]{neill2009, sullivan2010, lampeitl2010, kelly2010,
gupta2011, dandrea2011, childress2013, rigault2013, pan2014, kim19}. Following
the ``A+B'' model and the connection between SN stretch and host galaxy
properties, \cite{howell2007} first discussed the potential redshift drift of
the SN stretch distribution. In this paper we revisit this question
using the most recent SNe~Ia datasets.

In this paper, we take a step aside from the cosmological analyses to
probe the validity of our modeling of the SN population, which we claim to be
constituted of two age-populations \citep{rigault2013,rigault2015,rigault2020}:
one old and one younger, the former having on average lower lightcurve stretches
and being brighter after standardization. We use the correlation between the SN
age, as probed by the LsSFR, and the SN stretch to model the expected evolution
of the underlying SN stretch distribution as a function of redshift. This
modeling relies on three assumptions: (1) there are two distinct populations of
SNe~Ia; (2) the relative fraction of each of these populations as a function of
redshift follows the model presented in \cite{rigault2020} and (3) the
underlying distribution of stretch for each age sample is constant. This paper
aims at testing this specific model with datasets from the literature. 

We present in Section~\ref{sec:sample} the sample we are using for this
analysis, derived from the Pantheon catalog \citep{scolnic2018a}. We discuss the
importance of obtaining a ``complete'' sample, i.e.\ representative of the true
underlying SNe Ia distribution, and how we build one from the Pantheon sample.
We then present in Section~\ref{sec:modeling} our modeling of the distribution
of stretch and our results are presented in Section~\ref{sec:results}. In this
section, we test whether the SN stretch distribution evolves as a function of
redshift and if the aforementioned age model is in good agreement with this
evolution. We briefly discuss these results in the context of SN cosmology in
Section~\ref{sec:discussion} and we conclude in Section~\ref{sec:ccl}.

\section{Complete Sample Construction}\label{sec:sample}

The ideal SN~Ia sample for studying this question would be a very deep,
    large-area, volume-limited sample. This would capture the true underlying
    stretch distribution function, and we would then study how it evolves with
    redshift. No such sample exists, so first we must construct subsamples that
are as near to volume-limited as possible from existing high-redshift SN~Ia
samples.

\subsection{Applying redshift cuts}\label{ssec:cuts}

We base our analysis on the most recent comprehensive SNe~Ia compilation, the
Pantheon catalog from \cite{scolnic2018a}. A naive approach to test the SN
stretch redshift drift would be to simply compare the observed SN stretch
distributions in a few bins of redshift. In practice, however, differential
observational selection effects will affect the observed SN
stretch distributions. Indeed, because the observed SN~Ia magnitude correlates
with the lightcurve stretch (and color), the first SNe~Ia that a
magnitude-limited survey will miss are the lowest-stretch (and reddest) ones.
Consequently, if magnitude-related observational selection effects are
not accounted for, one might confuse true population drift with survey
properties, and conversely.

Assuming sufficient (and unbiased) spectroscopic follow-up for acquiring
SN types and host galaxy redshifts, the observation selection
effects of magnitude-limited surveys should be negligible below a given redshift
at which even the faintest normal SNe~Ia can be observed.
Aiding in the construction of nearly volume-limited subsamples is the
    fact that the SN~Ia population trails off towards fainter SNe~Ia. A
    complication is that complete spectroscopic follow-up has not always been
    the norm, as discussed below. In contrast, targeted surveys have highly
    complex observational selection functions and so are
    discarded from our analysis. High-redshift SN cosmology samples,
    such as Pantheon, are predominately from magnitude-limited surveys from
which volume-limited SN~Ia subsamples can be constructed.

We present in Fig.~\ref{fig:maglim} the lightcurve stretch and color of SNe~Ia
from the following surveys: PanStarrs \citep[PS1][]{rest2014}, the Sloan Digital
Sky Survey \citep[SDSS][]{frieman2008} and the SuperNovae Legacy Survey
\citep[SNLS][]{astier2006}. An ellipse in the \textsc{\texttt{SALT2.4}} $(x_1,
c)$ plane with $x_1 = \pm 3$ and $c = \pm 0.3$ encapsulates the full
parent distribution \citep{guy2007, betoule2014}; see also
\citet{bazin2011} and \citet{campbell2013} for similar contours, the second
using a more conservative $|c| \leq 0.2$ cut. Assuming the SN absolute magnitude
with $x_1=0$ and $c=0$ is $M_0=-19.36$ \citep{kessler2009,scolnic2014}, we can
derive the absolute standardized magnitude at maximum of light $M = M_0 - \alpha
x_1 + \beta c$ along the aforementioned ellipse given the standardization
coefficient $\alpha=0.156$ and $\beta=3.14$ from \cite{scolnic2018a}: the
faintest SN~Ia is that with $(x_1=-1.65, c=0.25)$ and an absolute standardized
magnitude at peak in Bessel $B$ band of $M^{t_0}_{\min} = -18.31$~mag. Since one
ought to detect this object typically 5 days before and a week after peak to
build a suitable lightcurve, the effective limiting standardized absolute
magnitude is approximately $M_{\lim} = -18.00$~mag. Hence, given the magnitude
limit $m_{\lim}$ of a magnitude limited survey, one can derive the maximum
redshift $z_{\lim}$ above which the faintest SNe~Ia will be missed using the
relation between apparent magnitude, redshift and absolute magnitude
$\mu(z_{\lim}) = m_{\lim} - M_{\lim}$.

\begin{figure}
    \centering
    \includegraphics[width=0.95\linewidth]{Article_figures/zmax_maglim_snls.pdf}
    \caption{\textsc{\texttt{SALT2.4}} stretch ($x_1$) and color ($c$)
        lightcurve parameters of SNe~Ia from the SDSS, PS1 and SNLS samples of
        the Pantheon catalog. The individual SNe are shown as blue dots. The
        ellipse $(x_1=\pm3, c=\pm0.3)$ is displayed, colored by the
        corresponding standardized absolute magnitude using the $\alpha$ and
        $\beta$ coefficients from \cite{scolnic2018a}. The grey diagonal lines
        represent the $(x_1, c)$ evolution for $m = m_{\lim}$, for $z$ between
        $0.50$ and $z=1.70$ using SNLS's $m_{\lim}$ of $24.8$~mag.}
    \label{fig:maglim}
\end{figure}

We will thus consider a set of cuts that will define a first fiducial
    sample, taking the limits as initially suggested by the previous
    completeness analysis. However, as this solution might be an overly
    simplified way to create a complete sample, e.g., because it ignores
    spectroscopic follow-up in efficiency for redshifts below $z_{lim}$, we
    also consider another set of cuts to define a so-called ``conservative''
    sample. The latter is smaller and therefore less statistically constraining,
    but also even less prone to observational selection effects. If the redshift
    drift is still significant in the conservative sample, it would be even more
    meaningful in a carefully-tailored selection-free sample. These samples are
    adequate for the goal of this study, which is to develop a first
    implementation of a model for drift in SN Ia properties. If fruitful, the
    sample selection can later be refined, if needed, with a more detailed model
of the observational selection, e.g., using the SNANA package \citep{SNANA}.

SNLS typically acquired SNe~Ia in the redshift range $0.4<z<0.8$; at these
redshifts, the rest-frame Bessel $B$ band roughly corresponds to the SNLS $i$
filter, which has a $5\sigma$ depth of
24.8~mag\footnote{\href{https://www.cfht.hawaii.edu/Science/CFHTLS/cfhtlsfinalreleaseexecsummary.html}{CFHT
final release website.}}. This converts to a $z_{\lim}=0.60$, in agreement with
\cite{neill2006}, \cite{perrett2010} and \cite{bazin2011}. Fig.~14 of
\citet[][see their Section~5]{perrett2010} suggests however a lower limit of
$z_{\lim}=0.55$. We will therefore use $z=0.60$ and $z=0.55$ as redshift
limits for the fiducial and conservative samples, respectively, for SNLS.

Similarly, PS1 observed SNe~Ia in the range $0.2<z<0.4$, their $g$-band
$5\sigma$ depth is 23.1~mag \citep{rest2014}, which yields $z_{\lim}=0.31$, in
agreement with, e.g., Fig.~6 of \cite{scolnic2018a}. If we were to be
    conservative, this figure would also suggest of a more stringent
$z_{\lim}=0.27$ cut; hence, we will use $0.31$ and $0.27$ for our fiducial and
conservative samples, respectively, for PS1.

In a similar redshift range, SDSS has a limiting magnitude of 22.5
\citep{dilday2008, sako2008}, which would lead to $z_{\lim}=0.24$. However, the
SDSS surveys had to contend with limited spectroscopic resources. As
discussed in \citet[][Section~2]{kessler2009}, during the first year of SDSS,
SNe~Ia with $r<20.5$~mag were favored for spectroscopic follow-up, corresponding
to a redshift cut at $0.15$. For the rest of the SDSS survey, additional
spectroscopic resources were available, and \cite{kessler2009} and
\cite{dilday2008} show a reasonable completeness up to $z_{\lim}=0.2$.
Based on this, we will use $z_{\lim}=0.20$ and $z_{\lim}=0.15$ for our
fiducial and conservative samples, respectively, for SDSS.

The sample selection is summarized in Table~\ref{tab:sample}, and the redshift
distribution of these three surveys is shown in Fig.~\ref{fig:cuts}. As
expected, the selected redshift limits are roughly located slightly
    before the peak of these histograms. In Section \ref{ssec:verify} we
validate that these redshift limits are effective for constructing nearly
volume-limited subsamples from samples that were initially more closely
magnitude-limited in their search or spectroscopic follow-up.

\begin{table}
    \centering
    \caption{Composition of the SNe~Ia dataset used in this analysis.
    Conservative cuts are indicated in parentheses. The SNf limit is set
by \cite{rigault2020}, see text.}
    \label{tab:sample}
    \begin{tabular}{l c c}
        \hline\hline
        Survey & $z_{\lim}$ & $N_{\mathrm{SN}}$ \\
        \hline
        SNf & 0.08 & 114 \\
        SDSS & 0.20 (0.15) & 167 (82)\\
        PS1 & 0.31 (0.27) & 160 (122)\\
        SNLS & 0.60 (0.55) & 102 (78)\\
        HST & -- & 26 \\
        \hline
        Total & -- & 569 (422) \\
        \hline
    \end{tabular}
\end{table}

\begin{figure}
    \centering
    \includegraphics[width=0.95\linewidth]{Article_figures/hist_surveys_cuts_55-cividis.pdf}
    \caption{\textit{From top to bottom}: Redshift histograms of SNe~Ia from the
        SDSS, PS1 and SNLS dataset respectively (data from Pantheon,
        \citealt{scolnic2018a}). The colored parts represent the distribution of
        SNe~Ia kept in our analysis for they are supposedly free from
    observational selection bias (see Section~\ref{sec:sample}). The
darker (resp. lighter) color responds to the conservative (resp. fiducial)
selection cut.}
    \label{fig:cuts}
\end{figure}

In addition, we use the SNe~Ia from the Nearby Supernova Factory
\citep[SNfactory,][]{aldering2002} published in \cite{rigault2020} and that have
been discovered from non-targeted searches (114 SNe~Ia, see their sections~3
and~4.2.2; SNe~Ia time series are published in \citealt{saunders2020}, see also
\citealt{aldering2020}). For this dataset, spectroscopic screening was
    done for candidates with $r \lesssim 19.5$; redshift cuts were then applied
    when selecting which SN Ia to follow, resulting in a redshift range of $0.02
< z < 0.09$, further reduced to $<0.08$ in \cite{rigault2020} for extracting
local host properties. These 114 SNfactory SNe~Ia are thus in the
volume-limited part of the survey (Aldering et al., in prep.), and are therefore
assumed to be a random sampling of the underlying SN population. The SNfactory
sample is particularly useful for studying SN property drift, as it enables us
to have a large complete SN~Ia sample at $z<0.1$. 

Finally, we include the HST sample from Pantheon \citep{strolger04}.
These high-redshift SNe are of great interest as they provide the
    greatest leverage for testing evolution. While at these redshifts the
    supernovae typing is challenging, the target classification was robust
    enough to include them within the cosmological analysis \citep{scolnic2018a}
    and we do not impose further cuts. Section~\ref{sec:results} highlights
    that, while compatible with it, our results are not dependent on the
    inclusion of this dataset.

We present the stretch distribution and redshift histogram of these five surveys
up to their respective $z_{\lim}$ in Fig.~\ref{fig:sample}. We observe here
that the fraction of low-stretch SNe (typically $x_1 < -1$) appears to decrease as
a function of redshift; this is confirmed in Fig.~\ref{fig:modelall}, in which
the evolution of the mean stretch is shown, with the data split in
redshift bins of regular sample size. We see that SNe~Ia at higher redshift have
on average larger stretch ($0.34 \pm 0.10$ at $z\sim0.65$) than those at lower
redshift ($-0.17\pm 0.10$ at $z\sim0.05$), suggesting that the underlying
stretch distribution is evolving with redshift.

\begin{figure}
    \centering
    \includegraphics[width=0.95\linewidth]{Article_figures/stretchs-cut_btw_hist_stac_75-lb-cividis.pdf}
    \caption{\textit{Bottom:} \textsc{\texttt{SALT2.4}} lightcurve stretch as a
        function of redshift for each survey considered in this analysis (see
        legend). Solid (resp.\ open) markers correspond to the conservative
        (resp.\ fiducial) redshift cuts. \textit{Top:} stacked redshift
    histograms in dark (resp.\ light) colors for the conservative (resp. \
fiducial) redshift cuts.}
    \label{fig:sample}
\end{figure}

\subsection{Testing the construction of a volume-limited sample}\label{ssec:verify}

In section~\ref{ssec:cuts}, we have built volume-limited samples from a set of
magnitude-limited ones, using simple redshift cuts. This simplified approach is
statistically sub-optimal, but should be enough to test our key question:
whether redshift evolution of stretch is compatible with the \cite{rigault2020}
model. However, the possibility remains that a complex observational selection
function related to spectroscopic follow-up efficiencies below our fiducial (or
even conservative) redshift cuts could still affect our sample, making it not
fully volume-limited; this would, in turn, bias our conclusion on the
astrophysical drift of the SNe~Ia population. We now look at this possibility.

To test for the existence of potential leftover observational selection biases
in our sample, we compare the stretch and color distributions of the SNe~Ia
originating from different datasets having overlapping redshift ranges: these
distributions should be similar if they reflect the underlying parent
population. We note that the redshift range has to be narrow enough so that any
drift would be negligible.

The two samples that overlap the most in redshift are PS1 and SDSS in the
redshift range $0.10 < z < 0.20$ (see Fig~\ref{fig:sample}). This overlapping
subsample consists of the 146 SNe~Ia at the high-redshift end of SDSS and thus
the most likely to be affected by residual observational selection effects (see
the corresponding discussion in section~\ref{ssec:cuts}). Over that same
redshift range, PS1 has 52 SNe~Ia, which are in the lowest redshift bins and
thus unlikely to have any observational selection issue. To identify potential
inconsistency between the PS1 and SDSS subsamples, Fig.~\ref{fig:distrib} (upper
panels) compares the stretch and color distribution of both these surveys. The
resulting Kolmogorov-Smirnov (KS) similarity test $p$-values ($p >10\%$) do not
support any inconsistency, in agreement with the visual impression from
Fig.~\ref{fig:distrib}.

We perform a similar analysis for PS1 and SNLS over the redshift range $0.20 < z
< 0.31$ (Fig.~\ref{fig:distrib}, lower panels), where the same conclusion can be
drawn: there is no substantial sign of discrepancy in the stretch and color
distributions between the low- and high-end of our fiducial SNLS and PS1
samples, respectively. Nonetheless, the small size of the SNLS dataset at $z <
0.31$ (12 SNe~Ia vs. 90 for PS1) limits the sensitivity of this test, and only a
strong deviation would be noticeable. Extending the redshift range to $0.20 < z
< 0.40$ (though we have no PS1 data above 0.3) allows to increase the SNLS
subsample to 31, yet the stretch $p$-values remains high (34\%) showing no sign
of inconsistency.

\begin{figure}
    \centering
    \includegraphics[width=0.95\linewidth]{Article_figures/both-cut_SDSS_SNLS_PS1.pdf}
    \caption{$x_1$ (left) and $c$ (right) distribution histograms of
            different surveys overlapping in redshift. \textit{Facing up}: SDSS
            and PS1 within the redshift range $0.10 < z < 0.20$; \textit{facing
            down}: PS1 and SNLS within the redshift range $0.20 < z < 0.31$.
            Error bars show the Poisson noise. Our stretch ``Base'' model is
            illustrated in orange at the mean redshift of the redshift ranges,
            $0.15$ and $0.25$, respectively. Kolmogorov-Smirnov test $p$-values
            are indicated on the top (resp. bottom) of each panel showing no
            sign that the SDSS and PS1 (resp. PS1 and SNLS) $x_1$ and $c$
            distributions are not drawn from the same underlying
    distributions}
    \label{fig:distrib}
\end{figure}

We finally highlight that the SNe~Ia color is more prone to
    observational selection effects than stretch as illustrated in
    Fig.~\ref{fig:maglim}; see also e.g., Fig.~3 of \cite{kessler2017}. Hence,
    since the comparison of color distributions shows no significant hint of
    leftover observational selection effect, it further supports our claim that
    our simple redshift-based selection criteria are sufficient to build the
    complete SNe~Ia samples required to test the redshift evolution of the
stretch distribution.

\section{Modeling the redshift drift}\label{sec:modeling}

\cite{rigault2020} presented a model for the evolution of the fraction of
younger and older SNe~Ia as a function of redshift following former work on
rates and delay time distributions \citep[e.g.,][]{mannucci2005,
scannapieco2005, sullivan2006, aubourg2008, childress2014, maozmannucci2014}.
In short, it was assumed that the number of ``young'' SNe~Ia follows the star
formation rate (SFR) in the Universe, while the number of ``old'' SNe~Ia follows
the number of Gyr-old stars in the Universe, i.e. the stellar mass (M$^*$).
Hence, if we denote $\delta(z)$ (resp. $\psi(z) = 1-\delta(z)$) the fraction of
young (resp. old) SNe~Ia in the Universe as a function of redshift, then the
ratio $\delta/\psi$ is expected to follow the evolution of the specific star
formation rate (SFR/M$^*$), which goes as $(1+z)^{2.8}$ until $z\sim2$
\citep[e.g.,][]{tasca2015}. Since $\delta(0.05) \sim \psi(0.05)$
\citep{rigault2013, rigault2020, wiseman2020}, in agreement with rate
expectations \citep{mannucci2006, rodney2014}, \cite{rigault2020} concluded that

\begin{equation}
    \label{eq:delta}
    \delta(z) = \left( K^{-1} \times (1+z)^{-2.8} +1 \right)^{-1}
\end{equation}
with $K=0.87$. This model is comparable to the evolution subsequently
predicted by \cite{childress2014} based on SN rates in galaxies depending on
their quenching time as a function of their stellar mass.

\subsection{``Base'' underlying stretch distribution}
\label{sec:basemodel}

To model the evolution of the full SN stretch distribution as a function of
redshift, given our aforementioned model of the evolution of the fraction of
younger and older SNe~Ia with cosmic time, we need to model the SN stretch
distribution for each age subsample. 

\cite{rigault2020} presented the relation between SN stretch and LsSFR
measurement, a progenitor age tracer, using the SNfactory sample. This relation
is shown in Fig.~\ref{fig:stretchlssfr} for the SNfactory SNe used in the
current analysis. Given the structure of the stretch-LsSFR scatter plot, our
model of the underlying SN~Ia stretch distribution is defined as follows:

\begin{figure*}
    \centering
    \includegraphics[width=0.8\linewidth]{Article_figures/model_base_hist.pdf}
    \caption{\textit{Main}: \textsc{\texttt{SALT2.4}} lightcurve stretch ($x_1$)
        as a function of the local specific star formation rate (LsSFR) for
        SNfactory SNe used in this analysis. The color corresponds to the
        probability, $p_y$, for the SNe~Ia to be young, i.e. to have
        $\log\mathrm{LsSFR} \geq -10.82$ \citep[see][]{rigault2020}.
        \textit{Right}: $p_y$-weighted histogram of the SN stretches, as well as
        the adjusted Base model; the younger and older population contributions
        are shown in purple and yellow, respectively.}
    \label{fig:stretchlssfr}
\end{figure*}

\begin{itemize}
    \item for the younger population (i.e., $\log(\mathrm{LsSFR})\geq-10.82$),
        the stretch distribution is modeled as a single normal distribution
        $\mathcal{N}(\mu_1, \sigma_1{}^2)$; 
    \item the older population (i.e., $\log( \mathrm{LsSFR})<-10.82$) stretch
        distribution is modeled as a bimodal Gaussian mixture $a\times
        \mathcal{N}(\mu_1, \sigma_1{}^2) + (1-a)\times \mathcal{N}(\mu_2,
        \sigma_2{}^2)$, where one mode is the same as for the young population,
        $a$ representing the relative influence of the two modes.
\end{itemize}

The stretch probability distribution function (pdf) of a given SN will be the
linear combination of the stretch distributions of these two population weighted
by its probability $y^i$ to be young (see Section~\ref{sec:basemodelapplied}).
But generally, the fraction of young SNe~Ia as a function of redshift is given
by $\delta(z)$ (see Eq.~\ref{eq:delta}) and therefore, our redshift drift model
of the underlying stretch distribution of SNe~Ia as a function of redshift $X_1(z)$ is
given by:
\begin{align}\label{eq:stretchz}
    X_1(z) = \delta(z)&\times \mathcal{N}(\mu_1,\sigma_1{}^2) + \nonumber \\
    (1-\delta(z))&\times \left[ a\times\mathcal{N}(\mu_1,\sigma_1{}^2) +
    (1-a)\times\mathcal{N}(\mu_2,\sigma_2{}^2) \right]
\end{align}
This is our Base drifting model.

\subsection{Comparison to data}\label{sec:basemodelapplied}

Given the probability $y^i$ that a given SN is young, and assuming our Base
model (see Section~\ref{sec:basemodel}), the probability to measure a
\textsc{\texttt{SALT2.4}} stretch $x_1^i$ with an error d$x_1^i$ is given by:
\begin{align}\label{eq:likelihoodsnf}
    \prob{x^i_1}{\vec{\theta}; \mathrm{d}x^i_1, y^i} =
    y^i & \times
    \mathcal{N}\left(x^i_1 \mid \mu_1, \sigma_1{}^2+\mathrm{d}x^i_1{}^2\right) +
    \nonumber\\
    (1-y^i) &\times \bigg[
    a \times \mathcal{N}\left(x^i_1 \mid \mu_1,
    \sigma_1{}^2+\mathrm{d}x^i_1{}^2\right) +
    \nonumber\\
    & (1-a) \times \mathcal{N}\left(x^i_1 \mid \mu_2,
    \sigma_2{}^{2}+\mathrm{d}x^i_1{}^2\right) \bigg]
\end{align}

The maximum-likelihood estimate of the 5~free parameters
$\vec{\theta}\equiv({\mu_1,\mu_2,\sigma_1,\sigma_2,a})$ of the model is obtained
by minimizing the following:
\begin{equation}\label{eq:likelihood}
    -2\ln(L) = -2 \sum_i \ln \prob{x_1^i}{\vec{\theta};
    \mathrm{d}x_1^i, y^i}.
\end{equation}

Depending on whether $y^i$ can be estimated directly from LsSFR measurements or
not, there are two ways to proceed, which we now discuss.

\subsubsection{With LsSFR measurements}\label{sec:modelpy}

For the SNfactory sample, we can readily set $y^i = p^i_y$, the probability to
have $\log(\textrm{LsSFR}) \geq -10.82$ (see Fig.~\ref{fig:stretchlssfr}), to
minimize Eq.~\ref{eq:likelihood} with respect to $\vec{\theta}$. Results on
fitting the SNf SNe with this model are shown Table~\ref{tab:modelresults} and
illustrated in Fig.~\ref{fig:modelall}.

\begin{table*}
    \centering
    \caption{Best fit values of the parameters for the Base stretch distribution
    model when applied to the SNfactory dataset only (114 SNe~Ia), the fiducial
569 SN~Ia sample or the conservative one (422).}
    \label{tab:modelresults}
    \begin{tabular}{lccccc}
        \hline\hline
        Sample & $\mu_1$ & $\sigma_1$
               & $\mu_2$ & $\sigma_2$
               & $a$ \\
        \hline
        SNfactory & $ 0.41 \pm 0.08$ & $0.55 \pm 0.06$
                  & $-1.38 \pm 0.10$ & $0.44 \pm 0.08$
                  & $ 0.48 \pm 0.08$ \\
        Fiducial & $ 0.37 \pm 0.05$ & $0.61 \pm 0.04$
                 & $-1.22 \pm 0.16$ & $0.56 \pm 0.10$
                 & $ 0.51 \pm 0.09$ \\
        Conservative & $ 0.38 \pm 0.05$ & $0.60 \pm 0.04$
                     & $-1.26 \pm 0.13$ & $0.53 \pm 0.08$
                     & $ 0.47 \pm 0.09$ \\
        \hline
    \end{tabular}
\end{table*}

\subsubsection{Without LsSFR measurements}\label{sec:modelnopy}

When lacking direct LsSFR measurements (i.e. $p_y^i$), we can extend the
analysis to non-SNfactory samples by using the redshift-evolution of the
fraction $\delta(z)$ of young SNe~Ia (Eq.~\ref{eq:delta}) as a proxy for the
probability of a SN to be young. This still corresponds to minimizing
Eq.~\ref{eq:likelihood} with respect to the parameters
$\vec{\theta}\equiv(\mu_1, \mu_2, \sigma_1, \sigma_2, a)$ of the stretch
distribution $X_1$ (Eq.~\ref{eq:stretchz}), but this time assuming $y^i =
\delta(z^i)$ for any given SN~$i$. 
\\
\\
For the rest of the analysis, we will therefore minimize Eq.~\ref{eq:likelihood}
using $p_y^i$ --- the probability for the SN $i$ to be young --- when available
(i.e. for SNfactory dataset), and $\delta(z^i)$ --- the expected fraction of
young SNe~Ia at the SN redshift $z^i$ --- otherwise.

Results of fitting this model to all the 569 (resp. 422) SNe from the fiducial
(resp. conservative) sample are given Table~\ref{tab:modelresults}, and the
predicted redshift evolution of mean stretch (expected $x_1$ given the
distribution of Eq.~\ref{eq:stretchz}) illustrated as a blue band in
Fig.~\ref{fig:modelall} accounting for parameters errors and their covariances.
We see in this figure that the measured mean SN~Ia stretch per redshift bins of
equal sample size closely follows our redshift drift modeling. This is indeed
what is expected if old environments favor low SN stretches
\citep[e.g.][]{howell2007} \textit{and} if the fraction of old SNe~Ia declines as
a function of redshift. See Section~\ref{sec:results} for a more quantitative
discussion.

\begin{figure*}
    \centering
    \includegraphics[width=0.7\linewidth]{Article_figures/stretchevol_all_vs_snf-mean.pdf}
    \caption{Evolution of the mean SN \textsc{\texttt{SALT2.4}} stretch ($x_1$)
        as a function of redshift. Markers show the stretch plain mean
        (with error estimated from scatter) measured in redshift bins of equal
        sample size, indicated in light gray at the bottom of each redshift bin.
        Full and light markers are used when considering the fiducial or the
        conservative samples, respectively. The orange horizontal line
        represents the mean stretch for the non-evolving Gaussian model
        (last line of Table~\ref{tab:comp}) on the fiducial sample. Best
        fits of our Base drifting model are shown as blue, dashed-blue and
        gray, when fitted on the fiducial sample, the conservative one or
        the SNfactory dataset only, respectively; all are compatible. The
        light-blue band illustrates the amplitude of the error (incl.
        covariance) of the best fit model when considering the fiducial
    dataset.}
    \label{fig:modelall}
\end{figure*}

\subsection{Alternative models}\label{sec:othermodel}

In Section~\ref{sec:basemodel}, we have modeled the underlying stretch
distribution following \cite{rigault2020}, i.e.\ as a single Gaussian for the
``young'' SNe~Ia and a mixture of two Gaussians for the ``old'' SNe Ia
population, one being the same as for the young population, plus another one for
the fast-declining SNe~Ia that seem to only exist in old local environments.
This is our so-called ``Base'' model. However, to test different modeling
choices, we have implemented a suite of alternative parametrizations that we
also adjust to the data following the procedure described in
Section~\ref{sec:modelnopy}. 

\cite{howell2007} used a simpler unimodal model per age category, assuming a
single normal distribution for each of the young and old populations. We thus
consider a ``Howell+drift'' model, with one single Gaussian per age group and
the $\delta(z)$ drift from Eq.~\ref{eq:delta}.

Alternatively, since we aim at probing the existence of an evolution with
redshift, we also test constant models by restricting the ``Base'' and
``Howell'' models to use a supposedly redshift-independent fraction $\delta(z)
\equiv f$ of young SNe; these models are hereafter labeled ``Base+constant'' and
``Howell+constant''.

We also consider another intrinsically non-drifting model, the functional form
developed for Beams with Bias Correction \cite[BBC,][]{scolnic2016,
kessler2017}, used in recent SN cosmological analyses
\cite[e.g.][]{scolnic2018a, descosmopaper2019, riess2016, riess2019} to account
for Malmquist biases. The BBC formalism assumes sample-based (hence
intrinsically non-drifting) asymmetric Gaussian stretch distributions:
$\mathcal{N}\left(\mu, \sigma_-{}^2\; \text{if} \;x_1<\mu,\; \text{else}
\;\sigma_+{}^2\right)$. The idea behind this sample-based approach is twofold:
(1) Malmquist biases are driven by survey properties and (2) since current
surveys cover limited redshift ranges, having a sample-based approach covers
some potential redshift evolution information \citep{scolnic2016, scolnic2018a}.
See further discussion concerning BBC in Section~\ref{sec:discussion}. 

Finally, for the sake of completeness, we also consider redshift-independent
pure and asymmetric Gaussian models. 

\section{Results}\label{sec:results}

We adjusted each of the models described above on both the fiducial and
conservative samples (cf. Section~\ref{sec:sample}); results are gathered
in Table~\ref{tab:comp}, and illustrated in Fig.~\ref{fig:mod_comp}. 

\begin{table*}
    \centering
    \caption{Comparison of the relative ability of each model to describe the
        data. For each considered model, we report if the model is drifting or
        not, its number of free parameters and, for both the fiducial and the
        conservative cuts, $-2\ln(L)$ (see Eq.~\ref{eq:likelihood}), the AIC and
        the AIC difference ($\Delta$AIC) between this model and the Base model
        used as reference for it has the lowest AIC.}
    \label{tab:comp}
    \begin{tabular}{ccc|ccc|ccc}
        \hline\hline
        & & & \multicolumn{3}{c}{Fiducial sample (569 SNe)}
            & \multicolumn{3}{|c}{Conservative sample (422 SNe)} \\
        Name & drift & $k$ &
        $-2\ln(L)$ & AIC & $\Delta$AIC & $-2\ln(L)$ & AIC & $\Delta$AIC\\
        \hline

        Base & $\delta(z)$ & 5
        & 1456.7 & 1466.7 & -- 
        & 1079.5 & 1089.5 & -- \\

        Howell+drift & $\delta(z)$ & 4
        & 1463.3 & 1471.3 & $-4.6$
        & 1088.2 & 1096.2 & $-6.7$ 
        \\

        Asymmetric & -- & 3
        & 1485.2 & 1491.2 & $-24.5$
        & 1101.3 & 1107.3 & $-17.8$ 
        \\

        Howell+const & $f$ & 5
        & 1484.2 & 1494.2 & $-27.5$
        & 1101.2 & 1111.2 & $-21.7$ 
        \\

        Base+const & $f$ & 6
        & 1484.2 & 1496.2 & $-29.5$
        & 1101.2 & 1113.2 & $-23.7$ 
        \\

        Per sample Asym. & per sample & 3$\times$5
        & 1468.2 & 1498.2 & $-31.5$
        & 1083.6 & 1113.6 & $-24.1$ 
        \\

        Gaussian & -- & 2
        & 1521.8 & 1525.8 & $-59.1$
        & 1142.6 & 1146.6 & $-57.1$ 
        \\
        \hline
    \end{tabular}
\end{table*}

\begin{figure}
    \centering
    \includegraphics[width=\linewidth]{Article_figures/mod_comp.pdf}
    \caption{$\Delta$AIC between ``Base'' model (reference) and other models
        (see Table~\ref{tab:comp}). Full and open blue markers correspond to
        models with and without redshift drift, respectively. Light markers show
        the results when the analysis is performed on the conservative sample
        rather than the fiducial one. Color-bands illustrate the validity of
        the models, from Acceptable ($\Delta\mathrm{AIC} > -5$) to Excluded
        ($\Delta\mathrm{AIC} < -20$), see text. According to the AIC, all
        non-drifting models (open symbols) are excluded to be as good a
        representation of the data as the Base (drifting) model.}
    \label{fig:mod_comp}
\end{figure}

Since the various models have different degrees of freedom, we use the Akaike
Information Criterion \citep[AIC, e.g.][]{burnham2004} to compare their ability
to properly describe the observations. This estimator penalizes extra degrees of
freedom to avoid over-fitting the data, and is defined as follow:
\begin{equation}
    \mathrm{AIC} = -2\ln(L) + 2k
\end{equation}
where $-2\ln(L)$ is derived by minimizing Eq.~\eqref{eq:likelihood}, and $k$ is
the number of free parameters to be adjusted. The reference model is the one
with the smallest AIC; in comparison to this model, the models with
$\Delta\mathrm{AIC}<5$ are coined acceptable, the ones with
$5<\Delta\mathrm{AIC}<20$ are unfavored, and those with $\Delta\mathrm{AIC}>20$
are deemed excluded. This roughly corresponds to 2, 3 and 5~$\sigma$ limits for
a Gaussian probability distribution. 

The best model (with smallest AIC) is the so-called Base model and thus is our
reference model; this is true both on the fiducial and conservative samples.
The Base model also has the smallest $-2\ln(L)$, making it the most likely
model even ignoring the over-fitting issue accounted for by the AIC formalism.

Furthermore, we find that redshift-independent stretch distributions are all
excluded as suitable descriptions of the data relative to the Base model. In
fact, the best non-drifting model (the Asymmetric one) has a very marginal
chance ($p \equiv \exp\left(\Delta\mathrm{AIC}/2\right) = 5\times10^{-6}$) to
describe the data as well as the Base model. This result is just a quantitative
assessment of qualitative facts clearly visible in Fig.~\ref{fig:modelall}: the
mean SN stretch per bin of redshift strongly suggests a significant redshift
evolution rather than a constant value, and this evolution is well described by
Eq.~\ref{eq:delta}.

Surprisingly, the sample-based Gaussian asymmetric modeling used by current
implementations of the BBC technique \citep{scolnic2016, kessler2017} has one of
the highest AIC value in our analysis (see Section~\ref{sec:results}). While its
$-2\ln(L)$ is the smallest of all redshift-independent models (but still $-11.5$
worse than the reference Base model), it is strongly penalized for requiring
15~free parameters ($\mu_0, \sigma_{\pm}$~for each of the 5~samples of the
analysis). Hence, its $\Delta\mathrm{AIC}<-20$, which could be interpreted as a
probability $p=2\times 10^{-7}$ of being as good a representation of the data
as the Base model.

We note that, when comparing models adjusted on individual subsamples rather
than globally, the Bayesian Information Criterion ($\mathrm{BIC} = -2\ln(L) +
k\ln(n)$, with $n$ the number of data points) might be better suited than AIC,
since it explicitly accounts for the fact that each subsample is fitted
separately: the sample-based model BIC is rightfully the sum of the BIC for each
sample. We find $\Delta\mathrm{BIC}=-48$, again refuting the sample-based
asymmetric Gaussian model as being as pertinent as the Base model.

In order to ensure that our results are not driven by the incompletely-modeled
HST subsample, we recompute $\Delta$AIC for each model excluding this dataset;
we find that it does not change $\Delta$AIC by more than few tenths. The
consistency of these values with those in Table~\ref{tab:comp} show that the HST
subsample does not drive our conclusions.

We report in Table~\ref{tab:bbc} our determination of $\mu_0$ and
$\sigma_{\pm}$ for each sample when implementing an asymmetric Gaussian
model, adjusted on the nominally selection-free samples using our fiducial cuts
(see Section~\ref{sec:sample}). We find our results in close agreement with
\cite{scolnic2016} for SNLS and SDSS and with \cite{scolnic2018a} for PS1, who
derived these model parameters using the full BBC formalism, using numerous
simulations to model the observational selection effects (see details
e.g., Section~3 of \citealt{kessler2017}). The agreement between our fit of the
asymmetric Gaussians on the supposedly selection-free part of the samples and
the results derived using the BBC formalism supports our approach to
constructing a sample with negligible observational selection
effects. If we were to use \cite{scolnic2016} and \cite{scolnic2018a} best fit
values of the $\mu_0, \sigma_{\pm}$ asymmetric parameters for SNLS, SDSS and
PS1, respectively, the $\Delta$AIC between our Base drifting model and the BBC
modeling would go even deeper from $-32$ to $-47$. We further discuss the
consequence of this result for cosmology in Section~\ref{sec:discussion}.
    
\begin{table}
    \centering
    \caption{Best-fit parameters for our sample-based asymmetric modeling of the
    underlying stretch distribution.}
    \label{tab:bbc}
    \begin{tabular}{ccccccc}
    \hline\hline
    Asymmetric & $\sigma_{-}$ & $\sigma_{+}$ & $\mu_0$ \\
    \hline
    SNfactory & 1.34 $\pm$ 0.13 & 0.41 $\pm$ 0.10 & 0.68 $\pm$ 0.15 \\
    SDSS & 1.31 $\pm$ 0.11 & 0.42 $\pm$ 0.09 & 0.72 $\pm$ 0.13 \\
    PS1 & 1.01 $\pm$ 0.11 & 0.52 $\pm$ 0.12 & 0.38 $\pm$ 0.16 \\
    SNLS & 1.41 $\pm$ 0.13 & 0.15 $\pm$ 0.13 & 1.22 $\pm$ 0.15 \\
    HST & 0.76 $\pm$ 0.36 & 0.79 $\pm$ 0.35 & 0.11 $\pm$ 0.44 \\
    \hline
    \end{tabular}
\end{table}
    
We also performed tests allowing the high-stretch mode of the old population to
differ from the young population mode, hence adding two degrees of freedom. The
corresponding fit is not significantly better, with a $\Delta$AIC of $-0.4$.
This reinforces our assumption that the young and old populations
indeed appear to share the same underlying high-stretch mode. Furthermore, one
might wonder whether a low-stretch mode might also exist in the young-population
(see Fig.~\ref{fig:stretchlssfr}). We test for this by allowing this
population to also be bimodal, finding the amplitude of such a
low-stretch mode to be compatible with~0 ($<2\%$) in this young population. More
generally, this raises the question of how well a given environmental
tracer (here LsSFR) traces the age. An analysis dedicated to this question will
be presented in Briday et al.\ in prep.

Finally, ignoring the LsSFR measurements --- available only for the SNfactory
dataset (see Section~\ref{sec:modeling}) --- reduces the significance of the
results presented in this section, as expected. Even so, non-drifting
models remain strongly disfavored. For instance, the best
fitting sample-based Gaussian asymmetric model is still
$\Delta\mathrm{AIC}<-10$ less representative of the data than our Base drifting
model.

\section{Discussion}\label{sec:discussion}

To the best of our knowledge, a SN~Ia stretch redshift drift modeling has never
been explicitly used in cosmological analyses, though Bayesian hierarchy
formalism such as UNITY \citep{rubin2015}, BAHAMAS \citep{shariff2016} or Steve
\citep{hinton2019} can easily allow it (see e.g., sections~1.3 and 2.5 of
\cite{rubin2015}). Not doing so is a second order issue for SN cosmology, as it
only affects the way one accounts for Malmquist bias. Indeed, as long as the
Phillips relation \citep{phillips1993} standardization parameter $\alpha$ is not
redshift dependent (a study beyond the scope of this paper, but see
e.g. \citealt{scolnic2018a}), the stretch-corrected SNe~Ia magnitudes used for
cosmology are blind to the underlying stretch distribution for complete samples.
However, surveys usually do have significant Malmquist bias for the upper half
of their SN redshift distribution. As a consequence, mismodeling of the
underlying stretch distribution will bias the SN magnitudes derived from such
surveys. 

Commonly used Malmquist bias correction techniques, such as the BBC-formalism,
assume per sample asymmetric Gaussian functions to model the underlying
stretch and color distributions. Yet, as shown in Section~\ref{sec:results},
such a sample-based distribution is excluded in comparison to our
drifting model. Then, contrary to what \citet[][Section~2]{scolnic2016}
and \citet[][Section~5.4]{scolnic2018a} have suggested --- i.e., that
traditional surveys span sufficiently limited redshift ranges
such that the per-sample approach accounts for implicit redshift drifts
--- a direct modeling of the redshift drift is more appropriate than a
sample-based approach. We add here that, as measurements of modern
surveys try to cover increasingly larger redshift ranges in order to reduce
calibration systematic uncertainties, this sample-based approach
becomes less valid, notably for PS1, DES and, soon, LSST.

We illustrate in Fig.~\ref{fig:bbc_pdf_ps1} the prediction difference in the
underlying stretch distribution between the per-sample asymmetric modeling and
our Base drifting model for the PS1 sample. Our model is bimodal and the
relative amplitude of each mode depends on the redshift-dependent fraction of
old and young SNe~Ia in the sample: the higher the fraction of old SNe~Ia (at
lower redshift), the higher the amplitude of the old-specific low-stretch mode.
This redshift dependency of the underlying stretch distributions is
shown running from blue to red in Fig.~\ref{fig:bbc_pdf_ps1} for the
redshift range covered by PS1. The observed $x_1$ histogram follows the
model we defined using the sum of individual underlying SN-redshift
distributions. As expected, the two modeling approaches differ mostly in the
negative part of the SN stretch distribution. The asymmetric Gaussian
distribution goes through the middle of the bimodal distribution,
over-estimating the number of SNe~Ia at $x_1\sim-0.7$ and under-estimating it at
$x_1\sim-1.7$ in comparison to our Base drifting model for typical PS1 SN
redshifts. This means that the SN bias-corrected standardized magnitude
estimated at a redshift suffering from by observational
selection effects would be biased by a mismodeling of the true
underlying stretch distribution.

\begin{figure}
    \centering
    \includegraphics[width=\linewidth]{Article_figures/bbc_comp_PS1_hist-nr.pdf}
    \caption{Distribution of the PS1 SN~Ia \textsc{\texttt{SALT2.4}} stretch
        ($x_1$) after the fiducial redshift limit cut (grey histogram). This
        distribution is supposed to be a random draw from the underlying stretch
        distribution. The green lines show the BBC model of this underlying
        distribution (asymmetric Gaussian). The full line (band) is our best fit
        (its error); the dashed line shows the \cite{scolnic2018a} result. The
        black line (band) shows our best fitted Base-modeling (its error, see
        Table \ref{tab:modelresults}) that includes redshift drift. For
        illustration, we show as colored (from blue to red with increasing
        redshifts) the evolution of the underlying stretch distribution as a
        function of redshift for the redshift range covered by PS1 data.}
    \label{fig:bbc_pdf_ps1}
\end{figure}

Assessing the amplitude of this magnitude bias for cosmology is beyond
the scope of this paper given the complexity of the BBC analysis. It would
require a full study using our Base model (Eq.~\ref{eq:stretchz}) in place of
the sample-based asymmetric modeling as part of the BBC simulations. However, we
already highlight that even if a non-drifting sample-based model could provide
comparable result in the volume-limited part of the various samples, these
models would differ when extrapolating to higher redshifts, precisely where the
underlying distribution will matter for correcting Malmquist biases.

In the era of modern cosmology, where we aim to measure $w_0$ at a
sub-percent level and $w_a$ with ten-percent precision
\citep[e.g.,][]{lsstpaper}, we stress that correct modeling of potential SN
redshift drift should be further studied and care should be taken when using
samples affected by observational selection effects.

\section{Conclusion}\label{sec:ccl}

We have presented an initial study of the drift of the underlying
SNe~Ia stretch distribution as a function of redshift. We built
effectively volume-limited SN~Ia subsamples from the Pantheon dataset
\citep[][SDSS, PS1 and SNLS]{scolnic2018a}, to which we added HST and SNfactory
data from \cite{rigault2020} for the high- and low-redshift bins. We only
considered the SNe that have been discovered in the redshift range of each
survey where observational selection effects are negligible, so that
the observed SNe~Ia stretches are a random sampling of the true underlying
distribution. This resulted in a fiducial sample of 569 SNe~Ia (422 SNe
when more conservative cuts were applied).

Following predictions made in \cite{rigault2020}, we introduced a redshift drift
model which depends on the expected fraction of ``young'' and ``old'' SNe~Ia as
a function of redshift, with each age population having its own
underlying stretch distribution.

In addition to this ``base'' model, we have studied various
distributions, including redshift independent models; we also studied the
prediction from a per-sample asymmetric Gaussian stretch distribution used, for
instance, by the Beams with Bias Correction Malmquist bias correction algorithm
\citep{scolnic2016, kessler2017}.

Our conclusions are the following:
\begin{enumerate}

    \item The underlying SN~Ia stretch distribution is significantly redshift
        dependent, as previously suggested by e.g.~\cite{howell2007}, in
        a way that observational selection effects alone cannot explain. This
        result is largely independent of details on each age-population model.
    
    \item Redshift-independent models are quantitatively excluded as
        suitable descriptions of the data relative to our Base model. This model
        assumes that: (1) the younger population has a unimodal Gaussian stretch
        distribution, while the older population stretch distribution is
        bimodal, one mode being the same as the young one; (2) the evolution of
        the relative fraction of younger and older SNe~Ia follows the prediction
        made in \cite{rigault2020}. This second result further supports
        the existence of both young and old SN~Ia populations, in agreement with
        rate studies \cite{mannucci2005, scannapieco2005, sullivan2006,
        aubourg2008}. 
        
    \item Models using survey-based asymmetric Gaussian distributions,
        e.g., as employed in the current implementation of BBC, are
        excluded as a good description of the data relative to our
        drifting model. Hence, the sample-based approach does not accurately
        account for redshift drift, a problem that will be exacerbated
        as surveys span increasingly larger redshift ranges. As a
    result, even if the necessary extra degrees of freedom might be acceptable
    given the large number of SNe~Ia in cosmological studies, extrapolating the
    SN property distributions from the volume-limited part of a survey to its
    Malmquist-biased magnitude-limited part would still be inaccurate
    because of the redshift evolution.

    \item Given the current dataset, we suggest the use of the following stretch
        population model as a function of redshift:
        \begin{align*}
        \label{eqconclusion:stretchz}
            X_1\left(z \right) =
            \delta(z)&\times\mathcal{N}(\mu_1,\sigma_1{}^2)\,+\nonumber\\
            (1-\delta(z))&\times \left[a\times\mathcal{N}(\mu_1,\sigma_1{}^2) +
            (1-a)\times\mathcal{N}(\mu_2,\sigma_2{}^2)\right]
            \tag{\ref{eq:stretchz}}
        \end{align*}
        with $a=0.51$, $\mu_1=0.37$, $\mu_2=-1.22$, $\sigma_1=0.61$,
        $\sigma_2=0.56$ (see Table~\ref{tab:modelresults}), and using the
        age-population drift model \begin{align*}
            \delta(z) & = \left( K^{-1} \times (1+z)^{-2.8} +1 \right)^{-1}
            \tag{\ref{eq:delta}}
        \end{align*}
        with $K=0.87$.
\end{enumerate}

In this paper, we considered a simple two-population Gaussian mixture
modeling. Additional data free from significant Malmquist bias would enable
us to refine it, as necessary. We note that samples at the low- and high-redshift
ends of the Hubble diagram would be particularly helpful for this drifting
analysis; fortunately this will soon be provided by the Zwicky Transient
Facility \citep[low-$z$,][]{bellm2019, graham2019}, and Subaru and SeeChange
SNe~Ia programs (high-$z$), respectively. 

The next step in this line of analysis will incorporate our model into
the SNANA framework \citep{SNANA}, both to more accurately account for
observational selection functions and to test the impact of our model on the
derivation of cosmological parameters; this study is underway.

\begin{acknowledgements}
    This project has received funding from the European Research Council (ERC)
    under the European Union's Horizon 2020 Research and Innovation program
    (grant agreement no 759194 - USNAC).
    This work was supported in part by the Director, Office of Science, Office
    of High Energy Physics of the U.S. Department of Energy under Contract No.
    DE-AC025CH11231.
    This project is partly financially supported by Région Rhône-Alpes-Auvergne.
\end{acknowledgements}

\bibliographystyle{aa}
\begin{thebibliography}{} 
% A

\bibitem[Abbott et al.(2019)]{descosmopaper2019} Abbott, T.~M.~C., Allam, S.,
Andersen, P., et al.\ 2019, \apjl, 872, L30

\bibitem[Aldering et al.(2002)]{aldering2002} Aldering, G., Adam, G., Antilogus,
P., et al.\ 2002, \procspie, 61

\bibitem[Aldering et al.(2020)]{aldering2020} Aldering, G., Antilogus, P.,
Aragon, C., et al.\ 2020, Research Notes of the American Astronomical Society,
4, 63

\bibitem[Astier et al.(2006)]{astier2006} Astier, P., Guy, J., Regnault, N., et
al.\ 2006, \aap, 447, 31

\bibitem[Aubourg et al.(2008)]{aubourg2008} Aubourg, {\'E}., Tojeiro, R.,
Jimenez, R., et al.\ 2008, \aap, 492, 631 

% B

\bibitem[Bazin et al.(2011)]{bazin2011} Bazin, G., Ruhlmann-Kleider, V.,
Palanque-Delabrouille, N., et al.\ 2011, \aap, 534, A43

\bibitem[Bellm et al.(2019)]{bellm2019} Bellm, E.~C., Kulkarni, S.~R., Graham,
M.~J., et al.\ 2019, \pasp, 131, 018002

\bibitem[Betoule et al.(2014)]{betoule2014} Betoule, M., Kessler, R., Guy, J.,
et al.\ 2014, \aap, 568, A22

\bibitem[Brout et al.(2019)]{brout2019} Brout, D., Scolnic, D., Kessler, R., et
al.\ 2019, \apj, 874, 150

\bibitem[Burnham \& Anderson(2004)]{burnham2004} Burnham, K., Anderson, D., \
2004, Sociological Methods \& Research, 33, 2

% C

\bibitem[Campbell et al.(2013)]{campbell2013} Campbell, H., D'Andrea, C.~B.,
Nichol, R.~C., et al.\ 2013, \apj, 763, 88

\bibitem[Childress et al.(2013)]{childress2013} Childress, M., Aldering, G.,
Antilogus, P., et al.\ 2013, \apj, 770, 108

\bibitem[Childress et al.(2014)]{childress2014} Childress, M.~J., Wolf, C., \&
Zahid, H.~J.\ 2014, \mnras, 445, 1898

% D

\bibitem[D'Andrea et al.(2011)]{dandrea2011} D'Andrea, C.~B., Gupta, R.~R.,
Sako, M., et al.\ 2011, \apj, 743, 172

\bibitem[Dilday et al.(2008)]{dilday2008} Dilday, B., Kessler, R., Frieman,
J.~A., et al.\ 2008, \apj, 682, 262

% E
% F

\bibitem[Feeney et al.(2019)]{feeney2019} Feeney, S.~M., Peiris, H.~V.,
Williamson, A.~R., et al.\ 2019, \prl, 122, 061105

\bibitem[Freedman et al.(2019)]{freedman2019} Freedman, W.~L., Madore, B.~F.,
Hatt, D., et al.\ 2019, \apj, 882, 34

\bibitem[Freedman et al.(2020)]{freedman2020} Freedman, W.~L., Madore, B.~F.,
Hoyt, T., et al.\ 2020, \apj, 891, 57. doi:10.3847/1538-4357/ab7339

\bibitem[Frieman et al.(2008)]{frieman2008} Frieman, J.~A., Bassett, B., Becker,
A., et al.\ 2008, \aj, 135, 338

% G

\bibitem[Graham et al.(2019)]{graham2019} Graham, M.~J., Kulkarni, S.~R., Bellm,
E.~C., et al.\ 2019, \pasp, 131, 078001

\bibitem[Gupta et al.(2011)]{gupta2011} Gupta, R.~R., D'Andrea, C.~B., Sako, M.,
et al.\ 2011, \apj, 740, 92

\bibitem[Guy et al.(2007)]{guy2007} Guy, J., Astier, P., Baumont, S., et al.\
2007, \aap, 466, 11

% H

\bibitem[Hamuy et al.(1996)]{hamuy1996} Hamuy, M., Phillips, M.~M., Suntzeff,
N.~B., et al.\ 1996, \aj, 112, 2391

\bibitem[Hamuy et al.(2000)]{hamuy2000} Hamuy, M., Trager, S.~C., Pinto, P.~A.,
et al.\ 2000, \aj, 120, 1479

\bibitem[Hinton et al.(2019)]{hinton2019} Hinton, S.~R., Davis, T.~M., Kim,
A.~G., et al.\ 2019, \apj, 876, 15

\bibitem[Howell et al.(2007)]{howell2007} Howell, D.~A., Sullivan, M., Conley,
A., et al.\ 2007, \apjl, 667, L37

% I

\bibitem[Ivezi{\'c} et al.(2019)]{lsstpaper} Ivezi{\'c}, {\v{Z}}., Kahn, S.~M.,
Tyson, J.~A., et al.\ 2019, \apj, 873, 111

% J

\bibitem[Jones et al.(2015)]{jones2015} Jones, D.~O., Riess, A.~G., \& Scolnic,
D.~M.\ 2015, \apj, 812, 3 1

\bibitem[Jones et al.(2018)]{jones2018} Jones, D.~O., Riess, A.~G., Scolnic,
D.~M., et al.\ 2018, \apj, 867, 108

\bibitem[Jones et al.(2018)b]{jones2018b} Jones, D.~O., Scolnic, D.~M., Riess,
A.~G., et al.\ 2018, \apj, 857, 51

\bibitem[Jones et al.(2019)]{jones2019} Jones, D.~O., Scolnic, D.~M., Foley,
R.~J., et al.\ 2019, \apj, 881, 19

% K

\bibitem[Kelly et al.(2010)]{kelly2010} Kelly, P.~L., Hicken, M., Burke, D.~L.,
et al.\ 2010, \apj, 715, 743

\bibitem[Kessler et al.(2009)]{kessler2009} Kessler, R., Becker, A.~C., Cinabro,
D., et al.\ 2009, \apjs, 185, 32

\bibitem[Kessler et al.(2009)]{SNANA} Kessler, R., Bernstein, J.~P., Cinabro,
D., et al.\ 2009, \pasp, 121, 1028

\bibitem[Kessler \& Scolnic(2017)]{kessler2017} Kessler, R., \& Scolnic, D.\
2017, \apj, 836, 56

\bibitem[Kim et al.(2018)]{kim18} Kim, Y.-L., Smith, M., Sullivan, M., et al.\
2018, \apj, 854, 24

\bibitem[Kim et al.(2019)]{kim19} Kim, Y.-L., Kang, Y., \& Lee, Y.-W.\ 2019,
Journal of Korean Astronomical Society, 52, 181

\bibitem[Knox \& Millea(2020)]{knox2019} Knox, L. \& Millea, M.\ 2020, \prd,
101, 043533. doi:10.1103/PhysRevD.101.043533

% L

\bibitem[Lampeitl et al.(2010)]{lampeitl2010} Lampeitl, H., Smith, M., Nichol,
R.~C., et al.\ 2010, \apj, 722, 566

% M

\bibitem[Mannucci et al.(2005)]{mannucci2005} Mannucci, F., Della Valle, M.,
Panagia, N., et al.\ 2005, \aap, 433, 807 

\bibitem[Mannucci et al.(2006)]{mannucci2006} Mannucci, F., Della Valle, M., \&
Panagia, N.\ 2006, \mnras, 370, 773 

\bibitem[Maoz et al.(2014)]{maozmannucci2014} Maoz, D., Mannucci, F., \&
Nelemans, G.\ 2014, \araa, 52, 107 


% N

\bibitem[Neill et al.(2006)]{neill2006} Neill, J.~D., Sullivan, M., Balam, D.,
et al.\ 2006, \aj, 132, 1126

\bibitem[Neill et al.(2009)]{neill2009} Neill, J.~D., Sullivan, M., Howell,
D.~A., et al.\ 2009, \apj, 707, 1449

\bibitem[Nordin et al.(2018)]{nordin2018} Nordin, J., Aldering, G., Antilogus,
P., et al.\ 2018, \aap, 614, A71

% O
% P

\bibitem[Pan et al.(2014)]{pan2014} Pan, Y.-C., Sullivan, M., Maguire, K., et
al.\ 2014, \mnras, 438, 1391

\bibitem[Perlmutter et al.(1999)]{perlmutter1999} Perlmutter, S., Aldering, G.,
Goldhaber, G., et al.\ 1999, \apj, 517, 565

\bibitem[Perrett et al.(2010)]{perrett2010} Perrett, K., Balam, D., Sullivan,
M., et al.\ 2010, \aj, 140, 518

\bibitem[Phillips(1993)]{phillips1993} Phillips, M.~M.\ 1993, \apjl, 413, L105

\bibitem[Planck Collaboration et al.(2020)]{planck2018} Planck
Collaboration, Aghanim, N., Akrami, Y., et al.\ 2020, \aap, 641, A6.
doi:10.1051/0004-6361/201833910

%\bibitem[Poulin et al.(2019)]{poulin2019} Poulin, V., Smith, T.~L., Karwal, T., et al.\ 2019, \prl, 122, 221301

% Q
% R

\bibitem[Reid et al.(2019)]{reid2019} Reid, M.~J., Pesce, D.~W., \& Riess,
A.~G.\ 2019, \apjl, 886, L27. doi:10.3847/2041-8213/ab552d

\bibitem[Rest et al.(2014)]{rest2014} Rest, A., Scolnic, D., Foley, R.~J., et
al.\ 2014, \apj, 795, 44

\bibitem[Riess et al.(1998)]{riess1998} Riess, A.~G., Filippenko, A.~V.,
Challis, P., et al.\ 1998, \aj, 116, 1009

\bibitem[Riess et al.(2009)]{riess2009} Riess, A.~G., Macri, L., Casertano, S.,
et al.\ 2009, \apj, 699, 539

\bibitem[Riess et al.(2016)]{riess2016} Riess, A.~G., Macri, L.~M., Hoffmann,
S.~L., et al.\ 2016, \apj, 826, 56

\bibitem[Riess et al.(2018)]{riess2018} Riess, A.~G., Casertano, S., Yuan, W.,
et al.\ 2018, \apj, 861, 126

\bibitem[Riess et al.(2019)]{riess2019} Riess, A.~G., Casertano, S., Yuan, W.,
et al.\ 2019, \apj, 876, 85

\bibitem[{Rigault et al.(2013)}]{rigault2013} Rigault, M., Copin, Y.,
Aldering, G., {et~al.} 2013, \aap, 560, A66

\bibitem[Rigault et al.(2015)]{rigault2015} Rigault, M., Aldering, G., Kowalski,
M., et al.\ 2015, \apj, 802, 20

\bibitem[Rigault et al.(2020)]{rigault2020} Rigault, M., Brinnel, V., Aldering,
G., et al.\ 2029, \aap, 644, A176

\bibitem[Rodney et al.(2014)]{rodney2014} Rodney, S.~A., Riess, A.~G., Strolger,
L.-G., et al.\ 2014, \aj, 148, 13 
  
\bibitem[Roman et al.(2018)]{roman2018} Roman, M., Hardin, D., Betoule, M., et
al.\ 2018, \aap, 615, A68

\bibitem[Rose et al.(2019)]{rose2019} Rose, B.~M., Garnavich, P.~M., \& Berg,
M.~A.\ 2019, \apj, 874, 32

\bibitem[Rubin et al.(2015)]{rubin2015} Rubin, D., Aldering, G., Barbary, K., et
al.\ 2015, \apj, 813, 137

\bibitem[Rubin \& Hayden(2016)]{rubin2016} Rubin, D., \& Hayden, B.\ 2016,
\apjl, 833, L30

% S

\bibitem[Sako et al.(2008)]{sako2008} Sako, M., Bassett, B., Becker, A., et al.\
2008, \aj, 135, 348

\bibitem[Saunders et al.(2020)]{saunders2020} Saunders, C., Aldering, G.,
Antilogus, P., et al.\ 2020, VizieR Online Data Catalog, J/ApJ/869/167

\bibitem[Scannapieco \& Bildsten(2005)]{scannapieco2005} Scannapieco, E., \&
Bildsten, L.\ 2005, \apjl, 629, L85 

\bibitem[Scolnic et al.(2014)]{scolnic2014} Scolnic, D., Rest, A., Riess, A., et
al.\ 2014, \apj, 795, 45

\bibitem[Scolnic \& Kessler(2016)]{scolnic2016} Scolnic, D., \& Kessler, R.\
2016, \apjl, 822, L35

\bibitem[Scolnic et al.(2018)]{scolnic2018a} Scolnic, D.~M., Jones, D.~O., Rest,
A., et al.\ 2018a, \apj, 859, 101

\bibitem[Scolnic et al.(2019)]{scolnicastro2020} Scolnic, D., Perlmutter, S.,
Aldering, G., et al.\ 2019, Astro2020: Decadal Survey on Astronomy and
Astrophysics, 2020, 270

\bibitem[Shariff et al.(2016)]{shariff2016} Shariff, H., Jiao, X., Trotta, R.,
et al.\ 2016, \apj, 827, 1

\bibitem[Strolger et al.(2004)]{strolger04} Strolger, L.-G., Riess, A.~G.,

\bibitem[Sullivan et al.(2006)]{sullivan2006} Sullivan, M., Le Borgne, D.,
Pritchet, C.~J., et al.\ 2006, \apj, 648, 868 

\bibitem[Sullivan et al.(2010)]{sullivan2010} Sullivan, M., Conley, A., Howell,
D.~A., et al.\ 2010, \mnras, 406, 782

% T

\bibitem[Tasca et al.(2015)]{tasca2015} Tasca, L.~A.~M., Le F{\`e}vre, O.,
Hathi, N.~P., et al.\ 2015, \aap, 581, A54

% U 
% V
% W

\bibitem[Wiseman et al.(2020)]{wiseman2020} Wiseman, P., Smith, M., Childress,
M., et al.\ 2020, \mnras, 495, 4040. doi:10.1093/mnras/staa1302

\bibitem[Wong et al.(2020)]{wong2019} Wong, K.~C., Suyu, S.~H., Chen, G.~C.-F.,
et al.\ 2020, \mnras, 498, 1420. doi:10.1093/mnras/stz3094

% X
% Y
% Z
\end{thebibliography}
\end{document}
